\documentclass[10pt,a4paper]{article}
\usepackage[utf8]{inputenc}
\usepackage[russian]{babel}
\usepackage{amsmath}
\usepackage{amsfonts}
\usepackage{amssymb}
\usepackage{graphicx}
\author{Александр Сухинин}
\title{Аналитическое чтение 2}
\begin{document}
\maketitle

\paragraph{Human-Centered Study of a Network Operations Center: Experience Report and Lessons Learned}
~

Данная статья об человекоориентированном исследовании сетевых операционных центров. Автор рассказывает о недостатке исследований таких центров, особенно с точки зрения взаимодействия людей. Во введении говориться об особенностях работы в NOC: о переработках, повышенной ответственности. В таких центрах высоко сотрудничество между работниками. Такие особенности накладывают ограничения на исследования, а именно: такие исследования не должны мешать повседневной работе сотрудников. Цель данной работы исследовать NOC. 

Такие центры работают 24/7. В них сотрудники работают в две смены: ночную и дневную. При этом, существуют люди ответственные за всю смену. 

Далее автор обосновывает выбор формата и людей для интервью. При том, нужно отметить, что автор не рассказывает никаких интересных выводов из этих интервью. Статья продолжается перечислением того, за чем наблюдал автор. При этом, опять же практически не вдаваясь в результаты наблюдений. Тем не менее, автор рассказывает о том, как происходит передача смены. Это один из самых тяжелых периодов в дне, так как при передаче смены необходимо рассказать о всех событиях произошедших за день и о текущих проблемах. Необходимо полностью ввести в курс дела новую смену.

В конце исследования автор делает следующие выводы:

\begin{enumerate}
\item Данная работа требует общественного сотрудничества
\item Данная работа сопряжена с проблемами передачи знаний, особенно, между сменами
\item Данная работа сопряжена с трудностями слежения за всеми событиями одновременно, особенно для ответственного по смене.
\end{enumerate}

Также автор извлек три урока из данного исследования:
\begin{enumerate}
\item Удачность продолжительного исследования по сравнению с интенсивным коротким
\item Удачность смешения методов изучения
\item Эффективность изучения от создания новых связей
\end{enumerate}

В целом данная статья видится достаточно неинтересной, так как рассказывает скорее не об операционных центрах, а о методах исследования операционных центров.

\end{document}