\documentclass{article}
\usepackage[utf8]{inputenc}
\usepackage[russian]{babel}
\usepackage[left=2cm,right=2cm,top=2cm,bottom=2cm,bindingoffset=0cm
]{geometry}
\usepackage{cmap}
\usepackage{fancyvrb}
\DefineShortVerb{\|}

\begin{document}

\begin{titlepage} \begin{center}

	\Large			
Санкт-Петербургский политехнический университет Петра Великого
			
	\vspace{0.2cm}	
Институт Информационных технологий и управления
		
	\vspace{2cm} \vfill \huge
Утилита для исследования сети и сканер портов Nmap
		
	\vfill 
	\begin{flushleft} \large \hangindent=8cm \hangafter=0
Выполнил: Сухинин А.А. гр. 53501/3 \hrulefill
			
Принял: Выглежанина К.Д. \hrulefill
	\end{flushleft}
		
	\vspace{2cm} \vfill \LARGE
2015 г.
		
\end{center} \end{titlepage}


\section{Цель работы}
~

Научиться сканировать хосты и порты, определять версии запущенных приложений.

\section{Ход работы}

\paragraph{Настройка}
~

Предварительно были скачаны образы Kali Linux и Metasploitable 2. Данные образы развернуты на виртуальных машинах, которые включены в режиме "Сетевой мост". В сети также присутствуют и другие компьютеры: один рабочий, три других ПК, домашний сервер. Шлюз - это роутер по адресу 191.168.1.1

\paragraph{Провести поиск активных хостов}
~

Поиск активных хостов можно произвести несколькими способами. Можно послать ICMP сообщение опрашивая все узлы либо попытаться просканировать популярные (1-1500) порты в диапазоне. Как правило, в современных сетях фильтруются ICMP пакеты, чтобы не предоставлять лишнюю информацию злоумышленнику и закрыть часть уязвимостей, таких как ping of death.

\begin{enumerate}
\item Стандартный ICMP ping

\begin{verbatim}
[*] exec: nmap -sn 192.168.1.*

Starting Nmap 6.47 ( http://nmap.org ) at 2015-05-24 11:58 EDT
Nmap scan report for router.asus.com (192.168.1.1)
Host is up (0.00052s latency).
MAC Address: 54:A0:50:83:A8:9C (Asustek Computer)
Nmap scan report for crazy_PC (192.168.1.25)
Host is up (0.000066s latency).
MAC Address: F4:6D:04:49:DC:FC (Asustek Computer)
Nmap scan report for 192.168.1.27
Host is up (0.044s latency).
MAC Address: 74:E5:43:65:15:F5 (Liteon Technology)
Nmap scan report for 192.168.1.35
Host is up (0.00021s latency).
MAC Address: 90:2B:34:DB:90:AD (Giga-byte Technology Co.)
Nmap scan report for crazy-mini (192.168.1.120)
Host is up (0.0046s latency).
MAC Address: C0:18:85:9E:54:0B (Hon Hai Precision Ind. Co.)
Nmap scan report for PODISH (192.168.1.132)
Host is up (0.017s latency).
MAC Address: 90:F6:52:6A:30:0D (Tp-link Technologies CO.)
Nmap scan report for kali (192.168.1.59)
Host is up.
Nmap done: 256 IP addresses (7 hosts up) scanned in 1.42 seconds
\end{verbatim}

\item Сканирование основных портов
\begin{verbatim}
[*] exec: nmap 192.168.1.*

Starting Nmap 6.47 ( http://nmap.org ) at 2015-05-24 12:01 EDT
Nmap scan report for router.asus.com (192.168.1.1)
Host is up (0.0024s latency).
Not shown: 996 closed ports
PORT     STATE SERVICE
53/tcp   open  domain
80/tcp   open  http
1723/tcp open  pptp
9998/tcp open  distinct32
MAC Address: 54:A0:50:83:A8:9C (Asustek Computer)

Nmap scan report for crazy_PC (192.168.1.25)
Host is up (0.000058s latency).
Not shown: 986 closed ports
PORT      STATE SERVICE
135/tcp   open  msrpc
139/tcp   open  netbios-ssn
445/tcp   open  microsoft-ds
554/tcp   open  rtsp
1025/tcp  open  NFS-or-IIS
1026/tcp  open  LSA-or-nterm
1027/tcp  open  IIS
1030/tcp  open  iad1
1045/tcp  open  fpitp
1048/tcp  open  neod2
1049/tcp  open  td-postman
2869/tcp  open  icslap
5357/tcp  open  wsdapi
10243/tcp open  unknown
MAC Address: F4:6D:04:49:DC:FC (Asustek Computer)

...
\end{verbatim}
\end{enumerate}

\paragraph{Определить открытые порты}
~

Для сканирования портов запустим уязвимую машину Metaspoitable 2.

Можно просканировать основные открытые порты командой: nmap 192.168.1.217

Либо указать весь диапазон портов: nmap 192.168.1.217 -p 1-65535

\begin{verbatim}
[*] exec: nmap 192.168.1.217 -p 1-65535

Starting Nmap 6.47 ( http://nmap.org ) at 2015-05-24 12:08 EDT
Nmap scan report for 192.168.1.217
Host is up (0.00018s latency).
Not shown: 65505 closed ports
PORT      STATE SERVICE
21/tcp    open  ftp
22/tcp    open  ssh
23/tcp    open  telnet
25/tcp    open  smtp
53/tcp    open  domain
80/tcp    open  http
111/tcp   open  rpcbind
139/tcp   open  netbios-ssn
445/tcp   open  microsoft-ds
512/tcp   open  exec
513/tcp   open  login
514/tcp   open  shell
1099/tcp  open  rmiregistry
1524/tcp  open  ingreslock
2049/tcp  open  nfs
2121/tcp  open  ccproxy-ftp
3306/tcp  open  mysql
3632/tcp  open  distccd
5432/tcp  open  postgresql
5900/tcp  open  vnc
6000/tcp  open  X11
6667/tcp  open  irc
6697/tcp  open  unknown
8009/tcp  open  ajp13
8180/tcp  open  unknown
8787/tcp  open  unknown
39142/tcp open  unknown
50756/tcp open  unknown
55303/tcp open  unknown
56967/tcp open  unknown
MAC Address: 08:00:27:C0:D5:A0 (Cadmus Computer Systems)

Nmap done: 1 IP address (1 host up) scanned in 7.73 seconds
\end{verbatim}


\paragraph{Определить версии сервисов}
~

Для этого необходимо добавить ключ -sV к предыдущему пункту.

\begin{verbatim}
[*] exec: nmap 192.168.1.217 -p "*" -sV

Starting Nmap 6.47 ( http://nmap.org ) at 2015-05-24 12:15 EDT
Nmap scan report for 192.168.1.217
Host is up (0.00049s latency).
Not shown: 4219 closed ports
PORT     STATE SERVICE     VERSION
21/tcp   open  ftp         vsftpd 2.3.4
22/tcp   open  ssh         OpenSSH 4.7p1 Debian 8ubuntu1 (protocol 2.0)
23/tcp   open  telnet      Linux telnetd
25/tcp   open  smtp        Postfix smtpd
53/tcp   open  domain      ISC BIND 9.4.2
80/tcp   open  http        Apache httpd 2.2.8 ((Ubuntu) DAV/2)
111/tcp  open  rpcbind     2 (RPC #100000)
139/tcp  open  netbios-ssn Samba smbd 3.X (workgroup: WORKGROUP)
445/tcp  open  netbios-ssn Samba smbd 3.X (workgroup: WORKGROUP)
512/tcp  open  exec?
513/tcp  open  login
514/tcp  open  shell?
1099/tcp open  rmiregistry GNU Classpath grmiregistry
1524/tcp open  shell       Metasploitable root shell
2049/tcp open  nfs         2-4 (RPC #100003)
2121/tcp open  ftp         ProFTPD 1.3.1
3306/tcp open  mysql       MySQL 5.0.51a-3ubuntu5
3632/tcp open  distccd     distccd v1 ((GNU) 4.2.4 (Ubuntu
4.2.4-1ubuntu4))
5432/tcp open  postgresql  PostgreSQL DB 8.3.0 - 8.3.7
5900/tcp open  vnc         VNC (protocol 3.3)
6000/tcp open  X11         (access denied)
6667/tcp open  irc         Unreal ircd
8009/tcp open  ajp13       Apache Jserv (Protocol v1.3)
8180/tcp open  http        Apache Tomcat/Coyote JSP engine 1.1
1 service unrecognized despite returning data. 
If you know the service/version, please submit the following fingerprint
at http://www.insecure.org/cgi-bin/servicefp-submit.cgi :
SF-Port514-TCP:V=6.47%I=7%D=5/24%Time=5561F938%P=i686-pc-linux-gnu
%r(NULL,
SF:2B,"\x01Couldn't\x20get\x20address\x20for\x20your\x20host\x20\(kali\)
SF:");
MAC Address: 08:00:27:C0:D5:A0 (Cadmus Computer Systems)
Service Info: Hosts:  metasploitable.localdomain, localhost,
irc.Metasploitable.LAN; OSs: Unix, Linux; CPE: cpe:/o:linux:linux_kernel
\end{verbatim}

\paragraph{Сохранить вывод утилиты в формате xml}
~

Для этого необходимо добавить к команде ключ -oX имя файла.

Файл, полученный в результате исполнения команды 
"nmap 192.168.1.217 -p"*" -sV -oX /home/nmap.xml" лежит в каталоге с
отчетом.

\paragraph{Изучить файлы nmap-services, nmap-os-db, nmap-service-probes}
~

Для удобства файл nmap-service-probes выкачан в каталог с отчетом.

\begin{enumerate}
\item nmap-service-probes

Перечислим основные директивы, используемые в файле.

\begin{enumerate}
\item Probe <протокол> <имя> q\dq<посылаемая строка>\dq

Где в качестве протокола может быть указать TCP или UDP, имя - любой набор английских символов, а между \dq \dq указывается строка, посылаемая на сервер.

\item match <название сервиса> <шаблон> [<версия>]

Сравнивает ответ с шаблоном, в случае соответствия завершает сопоставление.

\item softmatch  <название сервиса> <шаблон> [<версия>]

Аналогичен match, но не прекращает сопоставление в случае успеха.

\item totalwaitms  <миллисекунды>

Время ожидания
\end{enumerate}

\item nmap-os-db

Содержит набор отпечатков для каждой ОС представленных различными директивами.

Генерируются шесть пакетов специального вида, которые посылаются целевой машине с перерывом в 100 мс. Для получения результатов теста используются директивы SEQ, OPS, WIN и T1. Более подробную информацию можно получить по адресу http://nmap.org/book/osdetect-methods.html

\begin{enumerate}
\item SEQ - результаты последовательного анализа
\item OPS - флаги пакетов, полученных в ответ
\item WIN - размер окон
\item T1 - данные касательно ответа на первый пакет
\end{enumerate}

Также отпечаток может содержать директивы T2-T7 посылающие пакеты различного вида. Например, без указания флагов, с указанием флагов SYN, FIN, URG, PSH; а также пакеты другого вида. 

Кроме того, существует возможность тестировать указанный хост с помощью UDP пакетов (директива U1), а также множество других возможностей.

Модификация данного файла достаточно сложна и, как правило, производиться крайне редко.

\textbf{Пример отпечатка:}
\begin{verbatim}
# BT2700HGV DSL Router version 5.29.107.19
Fingerprint 2Wire BT2700HG-V ADSL modem
Class 2Wire | embedded || broadband router
CPE cpe:/h:2wire:bt2700hg-v
SEQ(SP=6A-BE%GCD=1-6%ISR=96-A0%TI=I%CI=I%II=I%SS=S%TS=A)
OPS(O1=M5B4NNSW0NNNT11%O2=M578NNSW0NNNT11%O3=M280W0NNNT11
%O4=M218NNSW0NNNT11%O5=M218NNSW0NNNT11%O6=M109NNSNNT11)
WIN(W1=8000%W2=8000%W3=8000%W4=8000%W5=8000%W6=8000)
ECN(R=Y%DF=Y%T=FA-104%TG=FF%W=8000%O=M5B4NNSW0N%CC=N%Q=)
T1(R=Y%DF=Y%T=FA-104%TG=FF%S=O%A=S+%F=AS%RD=0%Q=)
T2(R=N)
T3(R=N)
T4(R=Y%DF=Y%T=FA-104%TG=FF%W=0%S=A%A=Z%F=R%O=%RD=E44A4E43%Q=)
T5(R=Y%DF=Y%T=FA-104%TG=FF%W=0%S=Z%A=S+%F=AR%O=%RD=1F59B3D4%Q=)
T6(R=Y%DF=Y%T=FA-104%TG=FF%W=0%S=A%A=Z%F=R%O=%RD=1F59B3D4%Q=)
T7(R=N)
U1(DF=Y%T=FA-104%TG=FF%IPL=70%UN=0%RIPL=G%RID=G%RIPCK=G%RUCK=G%RUD=G)
IE(DFI=Y%T=FA-104%TG=FF%CD=S)
\end{verbatim}

\item nmap-services

Структура данного представлена в виде таблицы с тремя колонками.

Первая - имя сервиса.

Вторая - номер и тип порта.

Третья - как часто данный порт встречается.

\textbf{Фрагмент файла:}
\begin{verbatim}
systat	11/udp	0.000577	# Active Users
unknown	12/tcp	0.000063
daytime	13/tcp	0.003927
\end{verbatim}

\end{enumerate}

\paragraph{Выбрать пять записей из файла nmap-service-probes и описать их работу}
~

Для дополнительной наглядности рассмотрим распознанные сервисы на Metasploitable 2

\begin{enumerate}
\item Рассмотрим распознавание сервиса Samba

139/tcp  open  netbios-ssn Samba smbd 3.X (workgroup: WORKGROUP)

Найдем соответствующую строку в файле
\begin{verbatim}
match netbios-ssn m=^\0\0\0.\xffSMBr\0\0\0\0\x88..\0\0[-\w. ]*\0+@
\x06\0\0\x01\0\x11\x06\0.*(?:[^\0]|[^_A-Z0-9-]\0)((?:[-\w]\0){2,50})=s 
p/Samba smbd/ v/3.X/ i/workgroup: $P(1)/
\end{verbatim}
Как и было описано выше, строка состоит из директивы match, названия сервиса и шаблона. Шаблон состоит из регулярного выражения и строки для печати. К выражениям взятым в скобках, при печати можно обращаться как к параметрам. Данная директива сопоставляет ответ с регулярным выражением  
\begin{verbatim}
^\0\0\0.\xffSMBr\0\0\0\0\x88..\0\0[-\w. ]*\0+
@\x06\0\0\x01\0\x11\x06\0.*(?:[^\0]|[^_A-Z0-9-]\0)((?:[-\w]\0){2,50})
\end{verbatim}
При этом, выражение подставленное вместо указанного ниже может быть использовано в качестве параметра при печати. Остальные игнорируются т.к. внутри скобок указан знак вопроса. (Прим. w - весь алфавит и цифры)
\begin{verbatim}
((?:[-\w]\0){2,50})
\end{verbatim}
Последняя строка определяет результат при совпадении. Ключ p указывает имя продукта, ключ v - версию, а i - дополнительную информацию. При выводе дополнительной информации также используется вспомогательная функция P(), которая удаляет все непечатаемые символы из параметра.
\begin{verbatim}
p/Samba smbd/ v/3.X/ i/workgroup: $P(1)/
\end{verbatim}

\item Probe TCP NULL q||

Данная директива используется для тестирования TCP портов, ее название NULL. Видимо, это связано с тем, что она не передает никакой запрос серверу.

\item totalwaitms 6000

Данная строка означает, что максимальное время ожидания ответа равно шесть секунд.

\item Рассмотрим сопоставление для telnet
\begin{verbatim}
match telnet m|\xff\xfd\x18\xff\xfd \xff\xfd#\xff\xfd'$| p/Linux 
telnetd/ o/Linux/ cpe:/o:linux:linux_kernel/a
\end{verbatim}

Сравнивает ответ с последовательностью байт 0xff, 0xfd, 0x18, 0xff, 0xfd, 0xff, 0xfd, '\#', 0xff, 0xfd, ''', конец строки.

В случае успеха возвращает имя продукта Linux telnetd, ОС - Linux, cpe (Common platform enumeration) - o:linux:linux-kernel


\item Добавленные строчки:
\begin{verbatim}
Probe TCP HIYOU q|Hi, you!|

match simple tcp m|Hi!\r\nI'm Simple Server version ([0-9.]*)| 
p/Simple Server/ v/$P(1)/
\end{verbatim}

Первая строка посылает запрос на открытый TCP порт "Hi, you!".

В этом случае от сервера ожидается ответ:
\begin{verbatim}
Hi!
I'm Simple Server version X.X.X
\end{verbatim}

Из ответа извлекается версия и возвращается в качестве ответа.

\textbf{Пример использования nmap:}

\begin{verbatim}
[*] exec: nmap 192.168.1.25 -p 1879 -sV

Starting Nmap 6.47 ( http://nmap.org ) at 2015-05-24 17:09 EDT
Nmap scan report for crazy_PC (192.168.1.25)
Host is up (0.00018s latency).
PORT     STATE SERVICE      VERSION
1879/tcp open  SimpleServer Simple Server 1.0
MAC Address: F4:6D:04:49:DC:FC (Asustek Computer)

Service detection performed. Please report any incorrect results at 
http://nmap.org/submit/ .
Nmap done: 1 IP address (1 host up) scanned in 6.23 seconds
\end{verbatim}

\textbf{Пример использования nmap без изменений:}

\begin{verbatim}
[*] exec: nmap 192.168.1.25 -p 1879 -sV

Starting Nmap 6.47 ( http://nmap.org ) at 2015-05-24 17:19 EDT
Nmap scan report for crazy_PC (192.168.1.25)
Host is up (0.00024s latency).
PORT     STATE SERVICE VERSION
1879/tcp open  unknown
1 service unrecognized despite returning data. If you know the service/
version, please submit the following fingerprint at http://
www.insecure.org/cgi-bin/servicefp-submit.cgi :
SF-Port1879-TCP:V=6.47%I=7%D=5/24%Time=55624072%P=i686-pc-linux-gnu%r(Gene
SF:ricLines,5,"azaza")%r(GetRequest,5,"azaza")%r(HTTPOptions,5,"azaza")%r(
SF:RTSPRequest,5,"azaza")%r(RPCCheck,5,"azaza")%r(DNSVersionBindReq,5,"aza
SF:za")%r(DNSStatusRequest,5,"azaza")%r(Help,5,"azaza")%r(SSLSessionReq,
5,
SF:"azaza")%r(Kerberos,5,"azaza")%r(SMBProgNeg,5,"azaza")%r(X11Probe,
5,"az
SF:aza")%r(FourOhFourRequest,5,"azaza")%r(LPDString,5,"azaza")
%r(LDAPBindR
SF:eq,5,"azaza")%r(SIPOptions,5,"azaza")%r(LANDesk-RC,5,"azaza")
%r(Termina
SF:lServer,5,"azaza")%r(NCP,5,"azaza")%r(NotesRPC,5,"azaza")
%r(WMSRequest,
SF:5,"azaza")%r(oracle-tns,5,"azaza")%r(afp,5,"azaza")%r(kumo-server,
5,"az
SF:aza");
MAC Address: F4:6D:04:49:DC:FC (Asustek Computer)

Service detection performed. Please report any incorrect results at 
://nmap.org/submit/ .
Nmap done: 1 IP address (1 host up) scanned in 37.56 seconds
\end{verbatim}

\textbf{Сервер лежит в репозитории в каталоге programming}
\end{enumerate}


\paragraph{Выбрать один скрипт из состава Nmap и описать его работу}
~

В качестве скрипта, для рассмотрения был выбран скрипт перебора паролей ftp. Для удобства данный скрипт помещен в каталог с отчетом.

nmap предоставляет мощный движок для написания скриптов (NSE). Языком написания скриптов является LUA. nmap предоставляет обширную коллекцию скриптов, которая находится в поддиректории scripts.

Как и большинство исходных файлов, скрипт начинается с импорта зависимостей. Затем следуют его описание и комментарии к использованию. После указания автора, лицензии и категории скрипта начинается значимый код.

Оставшийся код можно разделить на три части:
Описание глобальных переменных, описание класса driver и использование движка перебора.

\begin{enumerate}
\item  Глобальные переменные

В этой части объявляются переменные указывающие используемый порт и  максимальный таймаут

\item Класс driver

Специального вида класс, с реализованным конструктором и методами connect, disconnect и login. В методах connect и disconnect производиться управление сокетом - установка и закрытие соединения с хостом указанным в конструкторе. Метод login осуществляет попытку авторизации. В данном методе, по открытому соединению последовательно передаются команды USER * и PASS * и далее анализируются полученные ответы. В случае, если авторизация прошла успешно, метод возращает true.

\item Функция action

В данной функции используется движок перебора паролей brute.Engine, которому в качестве параметров передаются имена пользователей и пароли, а также класс Driver.
\end{enumerate}

\paragraph{Просканировать виртуальную машину Metasploitable2 используя db nmap из состава metasploit-framework}
~

Предварительно необходимо включить postgresql и metasploit.

\begin{verbatim}
service postgresql start
service metasplot start
msfconsole
\end{verbatim}

Затем использовать любую команду из перечисленных выше, но вместо nmap использовать db nmap. Все результаты будут занесены в базу данных. Таким образом, db nmap позволяет повторно использовать результаты и экономить большое количество времени.

\begin{verbatim}
msf > db_nmap -sn 192.168.1.*
[*] Nmap: Starting Nmap 6.47 ( http://nmap.org ) at 2015-05-24 18:30 EDT
[*] Nmap: Nmap scan report for router.asus.com (192.168.1.1)
[*] Nmap: Host is up (0.0011s latency).
[*] Nmap: MAC Address: 54:A0:50:83:A8:9C (Asustek Computer)
[*] Nmap: Nmap scan report for crazy_PC (192.168.1.25)
[*] Nmap: Host is up (0.000062s latency).
[*] Nmap: MAC Address: F4:6D:04:49:DC:FC (Asustek Computer)
[*] Nmap: Nmap scan report for 192.168.1.27
[*] Nmap: Host is up (0.22s latency).
[*] Nmap: MAC Address: 74:E5:43:65:15:F5 (Liteon Technology)
[*] Nmap: Nmap scan report for crazy_server (192.168.1.35)
[*] Nmap: Host is up (0.00039s latency).
[*] Nmap: MAC Address: 90:2B:34:DB:90:AD (Giga-byte Technology Co.)
[*] Nmap: Nmap scan report for crazy-mini (192.168.1.120)
[*] Nmap: Host is up (0.11s latency).
[*] Nmap: MAC Address: C0:18:85:9E:54:0B (Hon Hai Precision Ind. Co.)
[*] Nmap: Nmap scan report for PODISH (192.168.1.132)
[*] Nmap: Host is up (0.13s latency).
[*] Nmap: MAC Address: 90:F6:52:6A:30:0D (Tp-link Technologies CO.)
[*] Nmap: Nmap scan report for 192.168.1.217
[*] Nmap: Host is up (0.00013s latency).
[*] Nmap: MAC Address: 08:00:27:C0:D5:A0 (Cadmus Computer Systems)
[*] Nmap: Nmap scan report for kali (192.168.1.59)
[*] Nmap: Host is up.
[*] Nmap: Nmap done: 256 IP addresses (8 hosts up) scanned in 2.43 seconds
\end{verbatim}

\paragraph{Исследовать различные этапы и режимы работы nmap с использованием утилиты Wireshark}
~

?

\section{Выводы}
~

В ходе данной работы были изучены основные возможности nmap. Определение активных хостов, сканирование портов, определение версий сервисов, дополнение определения версий сервисов, были рассмотрены основные файлы используемые для определения версий сервисов и ОС. В качестве примера - один скрипт перебора паролей. Также была рассмотрена версия db nmap сохраняющая результаты в БД для последующего применения. 

Инструмент nmap является мощным и гибким инструментом для сбора информации. При этом, не стоит забывать, что именно сбор информации определяет успех предстоящей атаки.
 
\end{document}