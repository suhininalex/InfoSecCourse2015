\documentclass{article}
\usepackage[utf8]{inputenc}
\usepackage[russian]{babel}
\usepackage[left=2cm,right=2cm,top=2cm,bottom=2cm,bindingoffset=0cm]{geometry}
\usepackage{amsmath}
\usepackage{cmap}
\usepackage{hyperref}

\begin{document}

\begin{titlepage} \begin{center}

	\Large			
Санкт-Петербургский политехнический университет Петра Великого
			
	\vspace{0.2cm}	
Институт Информационных технологий и управления
		
	\vspace{2cm} \vfill \huge
Программа для шифрования и подписи GPG,пакет Gpg4win	
		
	\vfill 
	\begin{flushleft} \large \hangindent=8cm \hangafter=0
Выполнил: Сухинин А.А. гр. 53501/3 \hrulefill
			
Принял: Выглежанина К.Д. \hrulefill
	\end{flushleft}
		
	\vspace{2cm} \vfill \LARGE
2015 г.
		
\end{center} \end{titlepage}


\section{Цель реботы} \label{Goal}
Научиться создавать сертификаты, шифровать файлы и ставить ЭЦП.

\section{Ход работы} \label{main}
\paragraph{Изучить документацию, запустить графическую оболочку Kleopatra}

~

GPG4Win - это установочный пакет для ОС Windows (2000/XP/2003/Vista/7), содержащий набор программ и руководств для подписи и шифрования документов. Их работа основывается на алгоритмах асимметричного шифрования. Пакет содержит два руководства: один для новичков и один для продвинутых пользователей.

Версия GPG4Win 2 включает следующие программы:
\begin{enumerate}
\item GnuPG - программа шифрования
\item Kleopatra - графическая оболочка
\item GNU Privacy Assistant (GPA) - графическая оболочка (альтернатива)
\item GnuPG for Outlook (GpgOL) - расширение Microsoft Outlook для шифрования и подписи сообщений
\item GPG Explorer eXtension (GpgEX) - расширение Windows Explore для шифрования и подписи файлов используя контекстное меню
\item Claws Mail - полноценная программа для работы с почтой, предлагающая хорошую поддержку GnuPG
\end{enumerate}

Основная идея асимметричного шифрования, в отличие от симметричного, заключается в создании пары различных ключей. Один для шифрования и один для расшифровки сообщений. Несмотря на то, что ключи связаны друг с другом, на практике оказывается невозможным нахождение одного ключа из другого.

В сочетании с хэшированием асимметричное шифрование часто используется для создания цифровой подписи. В этом случае, обычно создается и шифруется дайджест подписываемого документа, а второй ключ выкладывается в открытый доступ. При желании, любой человек может проверить, расшифровывается ли данный дайджест публичным ключом или нет. Возможность расшифровки открытым ключом подтверждает факт, что дайджест был зашифрован владельцем закрытого ключа. В тоже время, операция создания дайджеста на практике является необратимой. Это означает, что для существующего дайджеста нельзя создать новое подделанное сообщение.

Документация по GPG4Win достаточно объемная, поэтому более подробное знакомство с ней будет осуществлено по ходу выполнения работы.

\paragraph{Создать ключевую пару OpenPGP} 
~

Для создания ключевой пары необходимо выбрать пункт меню File - New Certificate - Create Personal OpenPGP key pair. Необходимо указать собственные данные, а также указать пароль для секретного ключа.

\paragraph{Экспортировать сетрификат}
~

Для экспорта сертификата необходимо вызвать контекстное меню на сертификате и выбрать пункт Export Certificates. Данный сертификат был экспортирован в корень репозитория.

Также необходимо экспортировать секретный ключ на случай непредвиденных обстоятельств. Для этого необходимо вызвать контекстное меню на сертификате и выбрать пункт Export Secret Keys. Секретный ключ, разумеется, в репозиторий не выкладывается.

\paragraph{Поставить ЭЦП на файл}
~

Поставить ЭЦП на файл можно двумя способами. Используя графическую оболочку Kleopatra это можно сделать через меню File - Sign/Encrypt Files. Далее необходимо выбрать файл и указать тип действия (шифрование, подпись, оба). Файлы lab1.tex и lab2.tex были подписаны. Их подписи расположены в том же каталоге что и сами файлы.

Существует более удобный способ подписи используя GpgEX. В проводнике достаточно в контекстном меню выбрать пункт Другие параметры GPG - Подписать. В этом случае нет необходимости выбирать действие и искать файл через диалоговое окно (да и вообще открывать графический интерфейс).

\paragraph{Получить чужой сертификат из репозитория}
~

Из каталога 
\url{https://github.com/vilegzhanina/InfoSecCourse2015/tree/master/%D0%9E%D1%82%D1%87%D0%B5%D1%82%D1%8B/01_LaTeX_Git_GPG}
были скачаны файлы karina.asc, myfirst.pdf, myfirst.pdf.sig. 


\paragraph{Импортировать сертификат, подписать его}
~

Используя меню File - Import Certificates графической оболочки Kleopatra импортируем сертификат. Далее вызываем контекстное меню на сертификате и выбираем пункт Certify Cerificate. Этим действием мы подтверждаем свое доверие к источнику. В случае, если все прошло успешно, сертификат перейдет во вкладку Trusted Certificates.

Кроме того, если до этого мы экспортировали секретный ключ, мы также можем его импортировать используя File - Import Certificates. В этом случае, главное не забыть выбрать в контекстном меню Change Owner Trust и указать, что это ваш сертификат, иначе Ваша заверка не будет учтена.

\paragraph{Проверить подпись}
~

Проверить подпись можно двумя способами: используя GpgEX и Kleopatra.
В первом случае достаточно вызвать контекстное меню на файле в проводнике и выбрать пункт расшифровать и проверить. В случае успешной проверки будет выведено сообщение The signature is valid and the certificate's validity is fully trusted. А также показаны данные подписавшего.  Для файла myfirst.pdf.sig данные выглядят следующим образом:

Signed on 2015-02-16 10:57 by k.vilegzhanina@gmail.com (Key ID: 0x391EA659).

The signature is valid and the certificate's validity is fully trusted.

~

Другой способ - это в графической оболочке Kleopatra выбрать пункт меню File - Decrypt/Verify File и указать файл подписи. Дальнейшие результаты и действия аналогичны.

\paragraph{Взять сертификат кого-либо из коллег, зашифровать и подписать для него какой-либо текст, предоставить свой сертификат, убедиться, что ему удалось получить открытый текст, проверить подпись}
~

Для примера был взят сертификат Баринова Д.С. и импортирован согласно предыдущим пунктам.

Далее был создан файл messageToBarinov.txt с текстом:

Как поживает твоя *.

Где * заменяет определенное слово. Данный текст был зашифрован для пользователей barinov и Karina Vilegzhanina <k.vilegzhanina@gmail.com>. Для этого в контекстном меню файла был выбран пункт зашифровать и подписать. Это же можно сделать при помощи меню Kleopatra.

Баринов Д.С. подтвердил получение и расшифровку сообщения и посмеялся.

\paragraph{Предыдущий пункт наоборот}
~

От Баринова Д.С. был получен файл ...

Сертификат импортирован в предыдущем пункте.

Для расшифровки был в контекстном меню был выбран пункт Расшифровать и проверить. Расшифровку можно провести также и при помощи графической оболочки Kleopatra выбрав соответствующий пункт меню File.

Текст сообщения следующий:

\paragraph{Используя GNU Privacy handbook (ссылка в материалах) потрени-роваться в использовании gpg через интерфейс командной строки,
без использования графических оболочек}
~

\begin{enumerate}
\item Создание пары ключей:

gpg --gen-key\\
\\
gpg (GnuPG) 1.4.13; Copyright (C) 2012 Free Software Foundation, Inc.\\
This is free software: you are free to change and redistribute it.\\
There is NO WARRANTY, to the extent permitted by law.\\
\\
Please select what kind of key you want:\\
   (1) RSA and RSA (default)\\
   (2) DSA and Elgamal\\
   (3) DSA (sign only)\\
   (4) RSA (sign only)\\
Your selection? 1\\
RSA keys may be between 1024 and 4096 bits long.\\
What keysize do you want? (2048)\\
Requested keysize is 2048 bits\\
Please specify how long the key should be valid.\\
         0 = key does not expire\\
      <n>  = key expires in n days\\
      <n>w = key expires in n weeks\\
      <n>m = key expires in n months\\
      <n>y = key expires in n years\\
Key is valid for? (0) 0\\
Key does not expire at all\\
Is this correct? (y/N) y\\
\\
You need a user ID to identify your key; the software constructs the\\ user ID\\
from the Real Name, Comment and Email Address in this form:\\
    "Heinrich Heine (Der Dichter) <heinrichh@duesseldorf.de>"\\
\\
Real name: Suhinin Alexandr\\
Email address: suhinin.alex@gmail.com\\
Comment:\\
You selected this USER-ID:\\
    "Suhinin Alexandr <suhinin.alex@gmail.com>"\\
\\
Change (N)ame, (C)omment, (E)mail or (O)kay/(Q)uit? o\\
You need a Passphrase to protect your secret key.\\
\\
We need to generate a lot of random bytes. It is a good idea to perform
some other action (type on the keyboard, move the mouse, utilize the
disks) during the prime generation; this gives the random number
generator a better chance to gain enough entropy.\\
.....+++++\\
.+++++\\
We need to generate a lot of random bytes. It is a good idea to perform
some other action (type on the keyboard, move the mouse, utilize the
disks) during the prime generation; this gives the random number
generator a better chance to gain enough entropy.\\
...+++++\\
+++++\\
gpg: key 52EE183C marked as ultimately trusted\\
public and secret key created and signed.\\
\\
gpg: checking the trustdb\\
gpg: 3 marginal(s) needed, 1 complete(s) needed, PGP trust model\\
gpg: depth: 0  valid:   2  signed:   1  trust: 0-, 0q, 0n, 0m, 0f, 2u\\
gpg: depth: 1  valid:   1  signed:   0  trust: 1-, 0q, 0n, 0m, 0f, 0u\\
pub   2048R/52EE183C 2015-05-24\\
      Key fingerprint = 8D1D 2045 011B 0F37 7FA4  A064 0E1E 6444 52EE\\ 183C\\
uid                  Suhinin Alexandr <suhinin.alex@gmail.com>\\
sub   2048R/F6FB8620 2015-05-24\\

\item Экспорт сертификата:

gpg --armor --export suhininalex@gmail.com\\
-----BEGIN PGP PUBLIC KEY BLOCK-----\\
Version: GnuPG v1.4.13 (MingW32)\\
\\
mQENBFVhpmMBCADoxGYGXVqwuf/X48f5bV91ds5susOX2mR77GmdBsidrF8ECwQq\\
aLAEEtHJqvvrBzNrD0++o+kgmb9IZZ5NGY67cR2fkgAxBOnm2ukgzho9iJJipd5W\\
0KSrsfx/6gLiopHBUNDjAOZRnzXT3AQ2zjMgFYjAEw/iexuMa5qqefaGFkUKWD2T\\
dGKQ8/C2zeUhFpUt1bk/0W1ISFSIEac2OCYI6E2eNu0k3jgF0DfCPPc3t7JpDgpu\\
XvRFv/GPY/S8eEJRRi8LcR9bXM3c9m0zI0JB7oRvpkmgBPjFmGo8Xehy7Qq051Mr\\
4yZkfiQiAGCRR2gGSIXsrPPJdDWmVKmQFqBZABEBAAG0JFN1aGluaW4gQWxleCA8\\
c3VoaW5pbmFsZXhAZ21haWwuY29tPokBOQQTAQgAIwUCVWGmYwIbDwcLCQgHAwIB\\
BhUIAgkKCwQWAgMBAh4BAheAAAoJENcnscQquDRqbGcH/0hRNiVcLlnqFBITvQnS\\
/pfG198Z4cPYneaacHUtNDX9ywYgTmjfDN9D93uzoWsOg32fTM0A5ZBhgLokUhvz\\
ZfVhePAez0ffK9Z9URb2PRe+HaCIzYXBmjfXMYT/7gsUSppQh66B6Rs3KdJc9dP3\\
9vw3ZiHjJaC5nqIIHJYOQkYVviuuVDsrZVN8WHbuS5+Nj5ea526dRS+plGr14McA\\
TN2IrvUQp0w2VCXkagWXwYye4qnpuuXITjXPUgGbSaSTz8JuPo436a4RVoGAFT02\\
4QqAKb9WEQMrNYujU9PE++VHMH3UIfXcJwAJ4h1Ri5cX8oWEkT/pMVP6RiSCciUf\\
j8w=\\
=7xTM\\
-----END PGP PUBLIC KEY BLOCK-----\\

\item Подпись:

gpg --detach-sign lab3.tex\\
\\
You need a passphrase to unlock the secret key for\\
user: "Suhinin Alex <suhininalex@gmail.com>"\\
2048-bit RSA key, ID 2AB8346A, created 2015-05-24\\

\item Проверка подписи:

gpg --verify lab2.tex.sig\\
\\
gpg: Signature made 05/24/15 13:25:27 using RSA key ID 2AB8346A\\
gpg: Good signature from "Suhinin Alex <suhininalex@gmail.com>"\\


\item Другие команды:

gpg --encrypt --recipient blake@cyb.org doc

gpg --decrypt doc.gpg

gpg --import blake.gpg

\end{enumerate}

\section{Выводы}
~

В ходе работы были опробованы основные возможности GPG4win. Создание пары ключей, сохранение и экспорт сертификата, импорт сертификатов и их проверка, подпись и проверка подписи файлов, шифрование и расшифровка файлов. В ходе работы был опробован графический интерфейс Kleopatra и расширение GpgEX. Некоторые основные команды были также опробованы в консоли.

Личные впечатления от работы с GPG4win остались положительные. Большинство интерфейсов понятны и просты в использовании. Особенно понравилось расширение GpgEX позволяющее подписывать, проверять, шифровать и расшифровывать файлы в несколько кликов. Графический интерфейс Kleopatra также удобен, но требует излишних действий (особенно надоедает постоянный поиск файлов в диалоговом окне). Таким образом, в будущем, я буду использовать Kleopatra для управления сертификатами, а GpgEX для работы с файлами.

Некоторые основные команды были опробованы также через консоль. Обычно я избегаю использования консоли без необходимости. Консольные команды обычно громоздки и сложны для запоминания,однако в этом случае они просты и интуитивно понятны. И, хотя, при наличии графической оболочки я бы все-равно отдал предпочтение GpgEX, тем не менее использование консоли также является удобным.

В качестве заключения следует еще раз отметить удобство и простоту GPG4win. С уверенностью можно сказать что данный пакет приложений полностью оправдывает свой девиз. Gpg4win - Cryptography for Everyone.
\end{document}