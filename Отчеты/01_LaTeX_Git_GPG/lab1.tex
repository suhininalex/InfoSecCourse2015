\documentclass{article}
\usepackage[utf8]{inputenc}
\usepackage[russian]{babel}
\usepackage[left=2cm,right=2cm,top=2cm,bottom=2cm,bindingoffset=0cm]{geometry}
\usepackage{amsmath}
\usepackage{cmap}


\begin{document}

\begin{titlepage} \begin{center}

	\Large			
Санкт-Петербургский политехнический университет Петра Великого
			
	\vspace{0.2cm}	
Институт Информационных технологий и управления
		
	\vspace{2cm} \vfill \huge
Система врестки Tex и расширения LaTex		
		
	\vfill 
	\begin{flushleft} \large \hangindent=8cm \hangafter=0
Выполнил: Сухинин А.А. гр. 53501/3 \hrulefill
			
Принял: Выглежанина К.Д. \hrulefill
	\end{flushleft}
		
	\vspace{2cm} \vfill \LARGE
2015 г.
		
\end{center} \end{titlepage}


\section{Цель реботы} \label{Goal}
Изучение принципов верстки Tex и расширения LaTex

\section{Ход работы} \label{main}
\paragraph{Создание минимального файла .tex в простом текстовом редакторе - преамбула, тело документа.}

~

Минимальный файл .tex состоит всего из определения типа документа, при помощи \textbackslash documentclass, а также из указания начала и окончания документа при помощи операторов \textbackslash begin\{document\} и end \{document\}

В более сложном варианте необходимо также указать локализацию, настроить отступы и указать дополнительные пакеты, которые будут использоваться в документе.

Более сложную преамбулу можно увидеть, например, в этом документе.

\paragraph{Компиляция в командной строке} 
~

Для компиляции достаточно ввести команду latex <имя файла>. В качестве результата получим dvi файл.

Для конвертации dvi файла в pdf необходимо использовать команду dvipdf, однако гораздо удобнее напрямую использовать pdflatex.

Для компиляции презантаций ипользуется команда xdvi.

\paragraph{Оболочка TexMaker, Быстрый старт, Быстрая сборка}
~

TexMaker - это удобная графическая оболочка для LaTex.

С его помощью можно быстро создать бланк документа, обратившись к помощнику и выбрав пункт меню "Быстрый старт". Далее пользователю будет предложено указать несколько основных настроек (тип документа, кодировка, используемые пакеты и т.п.) на основании которых будет сформирована начальная версия документа.

Также редактор предоставляет возможности полуавтоматического редактирования текста. Некоторые частоиспользуемые модификаторы вынесены в кнопки меню. Таким образом, изменить текст, например, сделав его жирным, можно также как и в более привычных редакторах. Для этого необходимо выделить текст и нажать на соответсвующую кнопку.

Главным преимуществом графического редактора является функция быстрой сборки. С ее помощью можно сразу же собрать документ и отобразить в соседней части экрана.

\paragraph{Создание титульного листа, нескольких разделов, списка, несложной формулы}
~

В качестве примера титульного листа, можно использовать первую страницу данного документа.

Также в этом документе созданы соответсвующие разделы.

Несложная формула и сприсок приведены ниже:
\begin{enumerate}
\item Это простой текст
\item Это простая формула №1: $$ x^2+y^2=r^2 $$
\item Это простая формула №2: $$ \frac{x^2}{y^2}=r^2 $$
\item Это простая формула №3: $$ \sqrt{x^2}=|x| $$
\item Это вложенный список
	\begin{itemize}
	\item Какой-то пункт
	\item Другой вложенный пункт
	\end{itemize}
\end{enumerate}

\paragraph{Понятие классов документов, подключаемых модулей}
~

\LaTeX -файл должен начинаться с команды \textbackslash documentclass. 

Данная команда задает стиль оформления документа. В стандартый пакет входят такие стили как article, book, report, letter, proc. \\
Класс article удобно применять для статей, report - для более крупных статей, разбитых на главы, класс book - для книг.

Кроме того, имеется в \LaTeX имеется возможность включения стилевых пакетов, доспускающих задание своих личных стилевых опций. Включение стилвого пакета осуществляется командой \textbackslash usepackage.

\paragraph{Верстка более сложных формул}
~

Матрица c индексами:
$$
A = \begin{pmatrix}
a_{11} & a_{12} & \cdots & a_{1n} \\
a_{21} & a_{22} & \cdots & a_{2n} \\         
\vdots & \vdots & \ddots & \vdots \\
a_{n1} & a_{n2} & \cdots & a_{nn}
\end{pmatrix}
$$

Система уравнений и греческие буквы:
\begin{equation}
X(\omega) = 
 \begin{cases}
   \omega = 3 &\text{se $\omega\in A$}\\
   x = 1250 &\text{se $\omega \in A^c$}
 \end{cases}
\end{equation}

При необходимости, нужную функциональность можно поискать в справочнике.

\section{Выводы}
~

В результате работы был составлен шаблон отчета для будущих работ, опробован редактор TexMaker и изучены основные конструкции \LaTeX .

После работы с \LaTeX остаются двоякие впечатления. С одной стороны, любые действия находящиеся в рамках стандартных, либо уже готовых шаблонах значительно упрощены. С другой стороны, изменения форматирования под себя является намного более неудобным по сравнению с MS Word. Кроме того, набор текстов не является интерактивным. Для получения результата, необходимо его в первую очередь скомпилировать, а затем просмотреть. Дополнительным недостатком является высокий порог вхождения, необходимо изучить достаточное количество литературы и просмотреть достаточное количество примеров для формирования достаточно тривиальных документов.

Следует отметить, что использование графических редакторов снижает недостатки, связанные с интерактивностью и сложностями при начальном ипользовании. 

Настоящее преимущество \LaTeX раскрывается при наприсании больших документов, либо при частом использовании однотипного шаблона.

Кроме того, \LaTeX лишен недостатков, связанных с неправильным отображением, либо изменением форматов от версии к версии. Также \LaTeX обладает значительно большими возможностями для верстки документов. И, таким образом, лучше подходит для высококачественной верстки документов.

В качестве вывода, хотелось бы отметить, что ниша \LaTeX - это профессиональная верстка документов. Повседневное использование \LaTeX возможно при условии отличного знания его функцональности и наличия необходимых шаблонов. В противном случае, временные затраты на создание документа возрастут неоправданно сильно.
\end{document}