\documentclass{article}
\usepackage[utf8]{inputenc}
\usepackage[russian]{babel}
\usepackage[left=2cm,right=2cm,top=2cm,bottom=2cm,bindingoffset=0cm]{geometry}
\usepackage{cmap}


\begin{document}

\begin{titlepage} \begin{center}

	\Large			
Санкт-Петербургский политехнический университет Петра Великого
			
	\vspace{0.2cm}	
Институт Информационных технологий и управления
		
	\vspace{2cm} \vfill \huge
Система контроля версий Git		
		
	\vfill 
	\begin{flushleft} \large \hangindent=8cm \hangafter=0
Выполнил: Сухинин А.А. гр. 53501/3 \hrulefill
			
Принял: Выглежанина К.Д. \hrulefill
	\end{flushleft}
		
	\vspace{2cm} \vfill \LARGE
2015 г.
		
\end{center} \end{titlepage}

\section{Цель работы}
~

Изучить систему контроля версий Git, освоить основные приемы работы
с ней.

\section{Ход работы}
~

Git - распределенная система контроля версий. Это подразумевает, что каждый разработчик хранит на собственной машине полноценную копию репозитория. Пока разработчик работает в рамках собственного репозиторя, все происходит в рамках обычной СКВ. Помимо этого, предусмотрены также команды для работы с удаленным репозиторием. Например, push для мержа изменений локального репозитория на удаленный и pull для обратной операции.

\paragraph{Изучить справку для основных команд}
~

Ниже приведен список самых частоиспользуемых команд:
\begin{enumerate}
\item git add

Добавление файла к репозиторию

\item git clone

Создание локальной копии репозитория

\item git push

Обновление удаленного репозитория

\item git pull

Обновление локального репозитория

\item git rm

Удаление файла из индекса

\item git diff

Отображение внесенных в файл изменений

\item git merge 

Слияние двух и более веток разработки в одну.
\end{enumerate}

\paragraph{Получить содержимое репозитория}
~

Для получения содержимого репозитория необходимо сделать его копию:

git clone https://github.com/suhininalex/InfoSecCourse2015.git

\paragraph{Добавить новую папку и первого файла под контроль версий}
~

git add /latex

git add lab1.tex

\paragraph{Зафиксировать изменения в локальном репозитории}
~

Перед тем как зафиксировать изменения удалим ненужные файлы: \\
git rm myfirst.pdf \\
...	\\
~

Фиксация: \\
git commit -a -m "Lab1" \\

\paragraph{Внести изменения в файл и посмотреть различия}
~

Для этого добавим в репозиторий файл текущего отчета, зафиксируем изменение, после чего сравним с измененной версией. Изменения можно просматривать между текущим состоянием и коммитом, между любыми двумя коммитами, также можно уточнять изменения какого файла именно нам интересны.

git diff 3ca89 39cd07 lab2.tex

\paragraph{Отменить локальные изменения}
~

git reset - -hard 3ca89

\paragraph{Внести изменения в файл и посмотреть различия}
~

git diff lab2.tex



\end{document}